\section{La bascule RS}


\subsection{Présentation}
\UPSTIdefinition[bascule]{Une bascule (ou verrou) est un circuit logique capable de maintenir les valeurs de ses sorties malgré les changements de valeurs d'entrées. 
Une bascule a donc un effet de mémoire.}

\UPSTIaRetenir{
	\begin{minipage}{.75\textwidth}
		\begin{UPSTIaCompleterEnv}
		Une bascule RS est une bascule à deux entrées de contrôle : $S$ (SET) et $R$ (RESET). Elle possède deux sorties \textit{$Q$} et \textit{$\overline{Q}$}. 
		Son comportement est le suivant : 
		\begin{itemize}
			\item Une mise à 1 de $S$ met la sortie $Q$ à 1
			\item Une mise à 1 de $R$ met la sortie $Q$ à 0
			\item $R=S=0 \rightarrow$ Etat mémoire : la sortie $Q$ maintient son état précédent.
			\item $R$ et $S$ à 1 est un état interdit
		\end{itemize}
		\vspace{1cm}
		\end{UPSTIaCompleterEnv}
		

	\end{minipage}
	\begin{minipage}{.25\textwidth}
		\begin{UPSTIaCompleterEnv}
			\begin{circuitikz}[]
			\draw (0,0) node[flipflop SR](RS){} (RS.bup)
			node[above]{Bascule RS};
			% \draw (RS.pin 1) -- ++(-1,0) node[and port,
			% anchor=out](AND1){}
			% (RS.pin 3) -- ++(-1,0) node[and port,
			% anchor=out](AND2){};
		\end{circuitikz}
		\end{UPSTIaCompleterEnv}
		\centering
	\end{minipage}

\begin{center}
	\begin{tabular}{cccc}
		\hline
		R & S & $Q$ 												& Remarque\\ \hline
		0 & 0 &{$Q$}	 		& {Mémoire}\\
		0 & 1 &{1}						& {Mise à 1}\\
		1 & 0 &0						& {Mise à 0}\\
		1 & 1 &0						& {Interdit}\\\hline
	\end{tabular}	
\end{center}
}

\begin{UPSTIactivite}
\UPSTIquestion{Compléter le chronogramme ci-dessous}
\begin{center}
	\begin{tikztimingtable}[
    timing/slope=0,         % no slope
    timing/coldist=2pt,     % column distance
    xscale=5,yscale=2, % scale diagrams
    semithick               % set line width
  ]

  $R$   					&  				  LLLLLLHLLL                   \\
  $S$                     &  				  LLHLHLLLLH     \\
  $Q$                     &  L \\
  $\overline{Q}$		&    H\\
\end{tikztimingtable}%
\end{center}
\end{UPSTIactivite}


\pagebreak

\begin{UPSTIactivite}[][Bascules ]
	
	\UPSTIquestion{Etablir une bascule RS en langage LADDER}
	\vspace{5cm}
\end{UPSTIactivite}
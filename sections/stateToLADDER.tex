\section{Traduction d'une machine à état en LADDER}
Au premier semestre, nous avions vu comment implémenter une machine à état à l'aide du langage C (avec l'utilisation de la structure \textit{switch ... case}. Ici, nous nous proposons d'implémenter une machine à état simple à l'aide du langage LADDER afin de la faire fonctionner sur un automate \textbf{LOGO}.

\UPSTIaRetenir{
	\begin{UPSTIaCompleterEnv}
		Une démarche pour traduire une machine à état en LADDER est la suivante :
		\begin{enumerate}
			\item \textbf{Dessiner} la machine à état
			\item Implémenter les transitions
			\item Implémenter les changements d'état
			\item Associer les actions aux différents états
		\end{enumerate}
		Par convention, les noms des transitions commenceront par un $t$ et les noms des variables par un $X$.
	\end{UPSTIaCompleterEnv}
}



\begin{UPSTIactivite}[2][][][\,][Application guidée][0]
	En suivant la méthode, traduire la machine à état à l'aide de bascules RS
	\vspace{15cm}
\end{UPSTIactivite}

\subsection{Les transitions}
Dans un premier temps, nous implémentons les transitions. Pour cela, voici les étapes à suivre :
\begin{enumerate}[label=(\alph*)]
	\item Déclarer une variable associée à cettre transition (exemple : $t_{0\rightarrow1}$ pour la transition de 0 vers 1)
	\item Écrire l'équation régissant cette transition
	\item Dessiner le réseau LADDER
\end{enumerate}

\UPSTIaRetenir{Une transition n'est franchie que si l'état précédent est actif et que la réceptivité associée est vérifiée.}

\UPSTIexemple{Pour l'auto-maintien, pour la transition depuis l'état 0 vers l'état 1 :
	\begin{enumerate}[label=(\alph*)]
		\item Utilisons la variable $t_{0\rightarrow1}$
		\item L'équation est : $t_{0\rightarrow1} = X_0 \cdot B_pM$
		\item Le réseau LADDER :

		      \begin{tikzpicture} [circuit plc ladder,thick,x=\ladderskip,y=\ladderskip]
			      \draw(0,0)
			      to [contact NO={info={$X_0$}}] ++(1,0)
			      to [contact NO={info={$B_pM$}}] ++(1,0) --++(1,0)
			      to [coil={info={$t_{0\rightarrow1}$}}] +(1,0) coordinate(laddertopright);

			      \ladderrungend{1.2}
			      % power rails
			      \draw let \p1=(laddertopright) in
			      (0,\y1+0.7\ladderskip) -- (0,\ladderskip)
			      (\x1,\y1+0.7\ladderskip) -- (\x1,\ladderskip);
		      \end{tikzpicture}
	\end{enumerate}}

\begin{UPSTIactivite}
	\UPSTIquestion{Dessiner le schéma LADDER correspondant à la transition 1 vers 0}

	\UPSTIaCompleter{
		\begin{enumerate}[label=(\alph*)]
			\item Utilisons la variable $t\_1-0$
			\item L'équation est : $t_{1\rightarrow0} = X_1 \cdot B_pA$
			\item Le réseau LADDER :
			      \begin{tikzpicture} [circuit plc ladder,thick,x=\ladderskip,y=\ladderskip]
				      \draw(0,0)
				      to [contact NO={info={$X_1$}}] ++(1,0)
				      to [contact NO={info={$B_pA$}}] ++(1,0) --++(1,0)
				      to [coil={info={$t_{0\rightarrow1}$}}] +(1,0) coordinate(laddertopright);
				      \ladderrungend{1.2}
				      % power rails
				      \draw let \p1=(laddertopright) in
				      (0,\y1+0.7\ladderskip) -- (0,\ladderskip)
				      (\x1,\y1+0.7\ladderskip) -- (\x1,\ladderskip);
			      \end{tikzpicture}
		\end{enumerate}}
	\vspace{3cm}

\end{UPSTIactivite}





\subsection{Activation des Etats}
Il faut à présent activer les état en à partir des transitions crées précédemment.

\UPSTIaRetenir{L'écriture d'un état en LADDER se fait en deux étapes :
	\begin{itemize}
		\item Mise à 1 (SET) de l'état à l'aide d'un \textit{OU} sur toutes les transitions \textbf{arrivant} sur celui-ci
		\item Mise à 0 (RESET) à l'aide d'un \textit{OU} sur toutes les transitions \textbf{partant} de celui-ci
	\end{itemize}}


Ainsi les réseaux associés à l'état 0 de l'auto-maintien sont donnés sur la Figure~\ref{fig:automaintienEtat0}


\begin{figure}[h!t]
  \centering
	\begin{tikzpicture} [circuit plc ladder,thick,x=\ladderskip,y=\ladderskip]

		\draw(0,0)
		to [contact NO={info={$t\_1-0$}}] ++(2,0) coordinate(laddercoil) -- ++(2,0)
		to [coil S={info={$X_0$}}] ++(1,0) coordinate(laddertopright);
		\draw(0,-1)
		to [contact NO={info={First cycle}}] ++(2,0) -- (laddercoil);
		\ladderrungend{2}
		\draw(0,0)
		to [contact NO={info={$t_{0\rightarrow1}$}}] ++(2,0) coordinate(laddercoil) -- ++(2,0)
		to [coil R={info={$X_0$}}] ++(1,0);
		\ladderrungend{1}
		% power rails
		\draw let \p1=(laddertopright) in
		(0,\y1+0.7\ladderskip) -- (0,\ladderskip)
		(\x1,\y1+0.7\ladderskip) -- (\x1,\ladderskip);
	\end{tikzpicture}
  \caption{Etat 0 de l'automaintien}
  \label{fig:automaintienEtat0}
\end{figure}


Le premier réseau active l'état 0 tandis que le deuxième le désactive.

\pagebreak
\begin{UPSTIactivite}
  \UPSTIquestion{Dessiner les réseaux LADDER implémentant le comportement de l'état 1}
  \vspace{5cm}
\end{UPSTIactivite}


\subsection{Associer les actions aux différents états}
Il s'agit ici d'activer les actionneurs lorsque l'état associé est actif.
Dans l'exemple de l'auto-maintien, la sortie $Q1$ devra-t-être active lorsque l'état 1 est actif.

\begin{UPSTIactivite}
  \UPSTIquestion{Dessiner un réseau LADDER permettant l'activation de la sortie lorsque l'état 1 est actif. 
  \vspace{5cm}}
\end{UPSTIactivite}

\subsection{Résumé de la méthode}

\UPSTIaRetenir[Méthode de traduction d'une machine à état en LADDER]{
  \begin{enumerate}
	\item \textbf{Dessiner} la machine à état
	\item Implémenter les transitions
	      \begin{enumerate}
		      \item Déclarer une variable associée à cettre transition (exemple : $t_{0\rightarrow1}$ pour la transition de 0 vers 1)
		      \item Écrire l'équation régissant cette transition
		      \item Dessiner le réseau LADDER
	      \end{enumerate}
	\item Implémenter les changements d'état
	      \begin{enumerate}
		      \item Mise à 1 (SET) de l'état à l'aide d'un \textit{OU} sur toutes les transitions \textbf{arrivant} sur celui-ci
		      \item Mise à 0 (RESET) à l'aide d'un \textit{OU} sur toutes les transitions \textbf{partant} de celui-ci
	      \end{enumerate}
	\item Associer les actions aux différents états
\end{enumerate}}

\element{machineAEtat}{
	\section{Machine à états}

	\UPSTIattention{Le cahier des charges de cette partie n'est pas compatible avec les programmes précédents. Après vérification des activités précédentes par un enseignant, réaliser cette partie dans un nouveau fichier}

	\UPSTIboiteCentrale{Cahier des charges}{
		\begin{enumerate}
			\item A l'appui sur le bouton poussoir, le voyant s'allume
			\item Lorsqu'une pièce métallique est détectée, le voyant clignote (utiliser le module de Asynchronious Pulse generator)
			\item Ensuite, si la tension du potentiometre vaut plus de 5V, le voyant s'éteint sur un appui sur le bouton rouge (NF)
			\item Le cycle recommence alors
		\end{enumerate}
	}

	\begin{UPSTIactivite}
		\begin{question}{act4-dessMachine}
			\evaluationProf[1]\\
			Dessiner la machine à état pour le comportement ci-dessus
			\vspace{-10pt}
		\end{question}


		\begin{question}{act4-ImpMachine}
			\evaluationProf[2]\\
			Implémenter la machine à état
			\vspace{-10pt}
		\end{question}
	\end{UPSTIactivite}
}
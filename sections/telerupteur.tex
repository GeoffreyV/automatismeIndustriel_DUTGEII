\subsection{Télérupteur}
Reprenons l'exemple du télérupteur. Un télérupteur a le comportement suivant :

\UPSTIboiteCentrale{Cahier des charges - Télérupteur}{
    \begin{itemize}
        \item Au départ la lumière est éteinte
        \item Un appui sur le bouton $BP$ allume la lumière
        \item Un nouvel appui sur le bouton $BP$ éteint la lumière et on reprend à l'état initial
    \end{itemize}
}

\begin{UPSTIactivite}
	\UPSTIquestion{Dessiner la machine à état d'un télérupteur avec puis sans utilisation de front montant}

	\begin{UPSTIaCompleterEnv}

		\begin{tikzpicture}[->,>=stealth',shorten >=1pt,auto,node distance=3cm,
				semithick]
			%\tikzstyle{every state}=[fill=none,draw=none,text=white]
			\node[initial,state] (A)              {M1};
			\node[state]         (B) [right of=A] {M2};
			\path (A) edge [bend left]  node {$\uparrow B_p$} (B)
			(B) edge [bend left]  node {$\uparrow B_p$} (A);
		\end{tikzpicture}
		% \begin{tikzpicture}[->,>=stealth',shorten >=1pt,auto,node distance=3cm,
		% 		semithick]
		% 	%\tikzstyle{every state}=[fill=none,draw=none,text=white]

		% 	\node[initial,state] (A)              {0 : Arrêt};
		% 	\node[state]         (B) [right of=A] {1 : Marche};
		% 	\node[state]         (C) [right of=B] {2 : Marche};
		% 	\node[state]         (D) [right of=C] {3 : Arrêt};

		% 	\path (A) edge [bend left]  node {$B_p$} (B)
		% 	(B) edge [bend left]  node {$\overline{B_p}$} (C)
		% 	(C) edge [bend left]  node {${B_p}$} (D)
		% 	(D) edge [bend left]  node {$\overline{B_p}$} (A);
		% \end{tikzpicture}
	\end{UPSTIaCompleterEnv}

    \UPSTIquestion{En suivant le protocole, implémenter cette machine à état à l'aide d'un circuit logique.}
\end{UPSTIactivite}
\section{Le texte structuré}
Le texte structuré est un langage de programmation proche du langage C.

Il consiste en une suite d'instructions séparées par un \textbf{;} . Ces instructions sont exécutées séquentiellement à chaque cycle automate.

\subsection{syntaxe du langage}
\subsubsection{opérateurs}
\begin{description}
  \item [Affection : ] On affecte une variable à l'aide de l'opérateur \textbf{:=}
    \begin{itemize}
      \item exemple : \textit{compteur := 10;}
    \end{itemize}
  \item [Test : ] Les opérateurs test (pour les structure IF par exemple) sont les suivants : \textbf{=} (égal) \textbf{>} (supérieur), \textbf{<} (inférieur), \textbf{<>} (différent).
\end{description}

\subsubsection{structure IF}
La structure IF permet de faire un test et de n'exécuter une portion de code que si une condition est vérifiée.
\begin{figure}[ht]
  \inputminted{C}{texteStructure/structureIF.c}
  \caption{Exemple d'une structure IF en texte structuré.}
\end{figure}

\subsection{Description d'un grafcet en texte structuré}

De plus en plus, les constructeurs préconisent l'utilisation du texte structuré pour décrire les grafcets.

La structure (Etape et transitions) d'un grafcet est décrit en texte structuré d'une manière analogue à la description d'une machine à état en langage C : A l'aide de la structure \textit{switch - case}. Un exemple est donné Figure~\ref{fig:grafcetSFC-ST}.

\begin{figure}[h]
  \begin{subfigure}{0.3\textwidth}
    \begin{tikzpicture}
\EtapeInit[0,0]{100}
\Transition[VX100]{100}
\Etape[VT100]{110}
\Transition{110}
\Etape[VT110]{120}
\Transition[VX120]{120}
\LienRetour{T120}{X100}
\Recept{T100}{$dcy$}
\ActionX{X110}{V+}
\Recept{T110}{fcV}
\ActionX{X120}{V-}
\Recept{T120}{$a_0$}
\end{tikzpicture}

    \caption{Grafcet en ladder}
  \end{subfigure}
  \hfill
  \begin{subfigure}{0.65\textwidth}
    \inputminted{C}{texteStructure/structureGrafcet.c}
    \caption{Structure d'un grafcet en texte structuré}
  \end{subfigure}
  \caption{Structure d'un grafcet en texte structuré}
  \label{fig:grafcetSFC-ST}
\end{figure}

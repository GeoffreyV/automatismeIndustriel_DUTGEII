\section{Le texte structuré}
Le texte structuré est un langage de programmation proche du langage C.

Il consiste en une suite d'instructions séparées par un \textbf{;} . Ces instructions sont exécutées séquentiellement à chaque cycle automate.

\subsection{syntaxe du langage}
\subsubsection{opérateurs}
\begin{description}
  \item [Affection : ] On affecte une variable à l'aide de l'opérateur \textbf{:=}
    \begin{itemize}
      \item exemple : \textit{compteur := 10;}
    \end{itemize}
  \item [Test : ] Les opérateurs test (pour les structure IF par exemple) sont les suivants : \textbf{=} (égal) \textbf{>} (supérieur), \textbf{<} (inférieur), \textbf{<>} (différent).
\end{description}

\subsubsection{structure IF}
La structure IF permet de faire un test et de n'exécuter une portion de code que si une condition est vérifiée.
\begin{figure}[ht]
  \inputminted{C}{texteStructure/structureIF.c}
  \caption{Exemple d'une structure IF en texte structuré.}
\end{figure}

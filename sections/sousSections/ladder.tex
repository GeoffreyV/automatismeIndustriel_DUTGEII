
Le langage LADDER fut conçu dans les années 1970 pour faire passer les électrotechniciens de la saisie de schémas électriques de systèmes à relais  à la programmation.
Il est donc simple et proche de la description de schémas électriques.
On le réserve à la description de fonctions combinatoires ou à des calculs simples.

Une ligne d'un programme LADDER est appelée réseau.

Un réseau est composé de contacts, de bobine et/ou de blocs.

\subsection{Réseaux LADDER simples}

\subsubsection{Les contacts}

Les contacts représentent des entrées logiques (TOR).

\UPSTIaRetenir{Il existe deux types de contacts :
\begin{description}
  \item [$\begin{array}{l}\begin{tikzpicture}[circuit plc ladder,thick,x=\ladderskip,y=\ladderskip]
  \draw(0,0) to [contact NO={info={a}}] ++ (1,0);
\end{tikzpicture}
 \end{array}$] \textbf{Contact normalement ouvert} :
  actif lorsque la variable associée est à l'état 1.
  \begin{itemize}
    \item [$\Rightarrow$] Il représente donc la variable $a$
  \end{itemize}
  \item [$\begin{array}{l}\begin{tikzpicture}[circuit plc ladder,thick,x=\ladderskip,y=\ladderskip]
  \draw(0,0) to [contact NC={info={a}}] ++ (1,0);
\end{tikzpicture}
 \end{array}$]  \textbf{Contact normalement fermé } actif lorsque la variable associée est à l'état 0.
  \begin{itemize}
    \item [$\Rightarrow$] Il représente donc la variable $\overline{a}$
  \end{itemize}
\end{description}
}

\subsubsection{Les bobines}

Les bobines représentent les sorties logiques (TOR) de l'API.

\UPSTIaRetenir{Une sortie S de l'automate est active lorsque la bobine associée $\begin{array}{l}\begin{tikzpicture}[circuit plc ladder,thick,x=\ladderskip,y=\ladderskip]
  \draw(0,0) to [coil={info={S}}] ++ (1,0);
\end{tikzpicture}
\end{array}$ est active.}


\subsubsection{Réseaux de base}

La traduction en réseau LADDER de l'équation $S = a$ est donc représentée sur la figure \ref{fig:ladAtoS}. L'équation en réseau LADDER de l'équation $S = \overline{a}$ est représentée sur la figure~\ref{fig:ladABartoS}

\begin{figure}[ht]
  \begin{subfigure}[b]{.49\textwidth}
  \centering
    \begin{tikzpicture}[circuit plc ladder, thick, x=\ladderskip, y=\ladderskip]
  \draw(0,0) to [contact NO={info={a}}] ++ (1,0) -- ++(1,0)
    to [coil={info={Sortie}}] + (1,0) coordinate(laddertopright);
    \ladderrungend{1.2}
    \draw let \p1=(laddertopright) in
    (0,   \y1+0.7\ladderskip) -- (0,    \ladderskip)
    (\x1, \y1+0.7\ladderskip) -- (\x1,  \ladderskip);
\end{tikzpicture}

    \caption{$S = a$}
    \label{fig:ladAtoS}
  \end{subfigure}
  %
  \begin{subfigure}[b]{.49\textwidth}
  \centering
    \begin{tikzpicture}[circuit plc ladder, thick, x=\ladderskip, y=\ladderskip]
  \draw(0,0) to [contact NC={info={a}}] ++ (1,0) -- ++(1,0)
    to [coil={info={Sortie}}] + (1,0) coordinate(laddertopright);
    \ladderrungend{1.2}
    \draw let \p1=(laddertopright) in
    (0,   \y1+0.7\ladderskip) -- (0,    \ladderskip)
    (\x1, \y1+0.7\ladderskip) -- (\x1,  \ladderskip);
\end{tikzpicture}

    \caption{$S = \overline{a}$}
    \label{fig:ladABartoS}
  \end{subfigure}

  \caption{Réseaux LADDER de base}
\end{figure}

Comme dans un circuit électrique, il est possible de programmer des équation combinatoires en langage LADDER. Les fonctions \textbf{ET} et \textbf{OU} sont représentée sur la figure~\ref{fig:equaLogiques}

\begin{figure}[ht]
  \begin{subfigure}[b]{.49\textwidth}
    \centering
    \begin{tikzpicture}[circuit plc ladder, thick, x=\ladderskip, y=\ladderskip]
  \draw(0,0) to [contact NO={info={a}}] ++ (1,0)
    to [contact NO={info={b}}] ++ (1,0) -- ++(1,0)
    to [coil={info={Sortie}}] + (1,0) coordinate(laddertopright);
    \ladderrungend{1.2}
    \draw let \p1=(laddertopright) in
    (0,   \y1+0.7\ladderskip) -- (0,    \ladderskip)
    (\x1, \y1+0.7\ladderskip) -- (\x1,  \ladderskip);
\end{tikzpicture}

    \caption{$S = a \text{ ET } b$}
    \label{fig:aETb}
  \end{subfigure}
  %
  \begin{subfigure}[b]{.49\textwidth}
    \centering 
    \begin{tikzpicture}[circuit plc ladder,thick,x=\ladderskip,y=\ladderskip]
  \draw(0,0) to [contact NO={info={a}}] ++ (1,0) coordinate(laddercoil) -- ++(2,0) to [coil={info={Sortie}}] ++ (1,0) coordinate(laddertopright);
  \draw(0,-1) to [contact NO={info={b}}] ++ (1,0) -- (laddercoil);

  \ladderrungend{2}
  \draw let \p1=(laddertopright) in
  (0,   \y1+0.7\ladderskip) -- (0,    \ladderskip)
  (\x1, \y1+0.7\ladderskip) -- (\x1,  \ladderskip);
\end{tikzpicture}

    \caption{$S = a \text{ OU } b$}
    \label{fig:aOUb}
  \end{subfigure}
  \caption{Equations logiques simples}
  \label{fig:equaLogiques}
\end{figure}

\begin{UPSTIactivite}
    \UPSTIquestion{Dessinez le réseau LADDER de l'équation $S = a + \overline{b}$}

    \UPSTIlignesACompleter[2]{\begin{center}
      \input{ladder_diagrams/ladCircuitaOubBar.tikz}
    \end{center}}

    \UPSTIquestion{Dessinez le réseau LADDER de l'équation $S = \overline{a} \cdot \overline{b}$}

    \UPSTIlignesACompleter[2]{\begin{center}
      \begin{tikzpicture}[circuit plc ladder, thick, x=\ladderskip, y=\ladderskip]
  \draw(0,0) to [contact NC={info={a}}] ++ (1,0)
    to [contact NC={info={b}}] ++ (1,0) -- ++(1,0)
    to [coil={info={Sortie}}] + (1,0) coordinate(laddertopright);
    \ladderrungend{2}
    \draw let \p1=(laddertopright) in
    (0,   \y1+0.7\ladderskip) -- (0,    \ladderskip)
    (\x1, \y1+0.7\ladderskip) -- (\x1,  \ladderskip);
\end{tikzpicture}

    \end{center}}

    \UPSTIquestion{Dessinez le réseau LADDER de l'équation $S = (a + b) \cdot \overline{b} \cdot c $}

    \UPSTIlignesACompleter[2]{\begin{center}
      \begin{tikzpicture}[circuit plc ladder,thick,x=\ladderskip,y=\ladderskip]
  \draw(0,0) to [contact NO={info={a}}] ++ (1,0) coordinate(laddercoil)
  to [contact NC={info={b}}] ++ (1,0) 
  to [contact NO={info={c}}] ++ (1,0)
  -- ++(2,0) to [coil={info={Sortie}}] ++ (1,0) coordinate(laddertopright);
  \draw(0,-1) to [contact NO={info={b}}] ++ (1,0) -- (laddercoil);

  \ladderrungend{2}
  \draw let \p1=(laddertopright) in
  (0,   \y1+0.7\ladderskip) -- (0,    \ladderskip)
  (\x1, \y1+0.7\ladderskip) -- (\x1,  \ladderskip);
\end{tikzpicture}

    \end{center}}


\end{UPSTIactivite}


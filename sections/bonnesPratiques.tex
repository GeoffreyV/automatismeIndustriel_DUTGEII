\section{Bonnes pratiques}
  Il est fortement conseillé de séparer la structure du programme de l'état des actionneurs.

  Cela signifie que vous devrez créer un grafcet sans les actions associées (en SFC ou en ST selon les TPs). Une tâche à part gérera les actions.

Un exemple du fichier action associé au grafcet de la figure~\ref{fig:grafcetSFC-ST} est donné ci-dessous (Figure~\ref{fig:actions}). Dans ce code, on trouve deux écritures possible pour un même comportement.

L'accès à l'état (actif ou non) d'un grafcet codé en SFC dépend des logiciels. La programmation de grafcet sera vue dans les différents TPs sur les parties opératives.

\begin{figure}[ht]
  \inputminted{C}{texteStructure/actions.c}
  \caption{Fichier actions associé au grafcet de la Figure~\ref{fig:grafcetSFC-ST}}
  \label{fig:actions}
\end{figure}

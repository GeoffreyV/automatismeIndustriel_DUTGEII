%%%%%%%%%%%%%%%%%%%%%%%%%%%%%%%%%%%%%%%%%%%%%%%%
% E.Pinault-Bigeard - e.pinault-bigeard@upsti.fr
% http://s2i.pinault-bigeard.com
% CC BY-NC-SA 2.0 FR - http://creativecommons.org/licenses/by-nc-sa/2.0/fr/
%%%%%%%%%%%%%%%%%%%%%%%%%%%%%%%%%%%%%%%%%%%%%%%%
\documentclass[11pt]{article}
%%%%%%%%%%%%%%%%%%%%%%%%%%%%%%%%%%%%%%%%%%%%%%%%
% Package UPSTI_Document
%%%%%%%%%%%%%%%%%%%%%%%%%%%%%%%%%%%%%%%%%%%%%%%%

\usepackage{import}
%
%%%%%%%%%%%%%%%%%%%%%%%%%%%%%%%%%%%%%%%%%%%%%%%%
% Package UPSTI_Document
%%%%%%%%%%%%%%%%%%%%%%%%%%%%%%%%%%%%%%%%%%%%%%%%
\usepackage{subcaption}
\usepackage[usenames, svgnames, dvipsnames]{xcolor}
\usepackage{UPSTI_Document}
\usepackage{pgfplots}
\usepackage{import}
\definecolor{darkspringgreen}{rgb}{0.09, 0.45, 0.27}

\newcommandx*{\dessinRepereFigGeo}[5][1=\vx{},2=\vy{},3=\vz{},4=,5=0]
	{
		\draw [->,very thick] (0,0) -- (1,0) ;
		\draw [->,very thick] (0,0) -- (0,1) ;
    \fill[white] (0,0) circle (0.13);
    \draw [->,very thick] (0,0) circle (0.13);
    \ifnumequal{#5}{0} {% z vers nous
      \fill[black] (0,0) circle (0.03);
      \draw [->,thick] (0,0) circle (0.04);
    }{% z vers la feuille
  		\begin{scope} [rotate=45]
  			\draw [-,thick] (0,-0.12) -- (0,0.12) ;
  			\draw [-,thick] (-0.12,0) -- (0.12,0) ;
  		\end{scope}
    }
		\draw [anchor=north west] (1.1,0) node {${#1}$};
		\draw [anchor=south west] (0,1.1) node {${#2}$};
		\draw [anchor=north east] (-0.1,0) node {${#3}$};
		\draw [anchor=north west] (-0.1,-0.1) node {${#4}$};
	}

	\usepackage{array}
	\newcolumntype{L}[1]{>{\raggedright\let\newline\\\arraybackslash\hspace{0pt}}m{#1}}
	\newcolumntype{C}[1]{>{\centering\let\newline\\\arraybackslash\hspace{0pt}}m{#1}}
	\newcolumntype{R}[1]{>{\raggedleft\let\newline\\\arraybackslash\hspace{0pt}}m{#1}}

	\usepackage{pifont}% http://ctan.org/pkg/pifont
\newcommand{\cmark}{\color{green}\ding{51}}%
\newcommand{\xmark}{\color{red}\ding{55}}%
\newcommand{\fmark}{\ding{229}}%
\newcommand{\itemc}{\item[\cmark]}%
\newcommand{\itemx}{\item[\xmark]}%
\newcommand{\itemf}{\item[\fmark]}%

\usepackage{tikz-timing}
\usepackage{circuitikz}
%---------------------------------%
% Paramètres du package
%---------------------------------%

% Version du document (pour la compilation)
% 1: Document prof
% 2: Document élève
% 3: Document à publier
\newcommand{\UPSTIidVersionDocument}{2}

% Classe
% 1: PTSI				6: PSI*			11: TSI2		16: Spé
% 2: PT	(par défaut)	7: MPSI			12: ATS
% 3: PT*				8: MP			13: PC
% 4: PCSI				9: MP*			14: PC*
% 5: PSI				10: TSI1		15: Sup
%\newcommand{\UPSTIidClasse}{2}



% Matière
% 1: S2I (par défaut)    2: IPT     3: TIPE
% 6: Vie au lycée
\newcommand{\UPSTIvariante}{5}
\newcommand{\UPSTIidMatiere}{0}
\newcommand{\UPSTIintituleMatiere}{Automatique}
\newcommand{\UPSTIsigleMatiere}{Autom}
% Type de document
% 0: Custom*				7: Fiche Métho de			14: Document Réponses
% 1: Cours (par défaut)		8: Fiche Synthèse    		15: Programme de colle
% 2: TD     				9: Formulaire
% 3: TP						10: Memo
% 4: Colle					11: Dossier Technique
% 5: DS						12: Dossier Ressource
% 6: DM						13: Concours Blanc
% * Si on met la valeur 0, il faut décommenter la ligne suivante:
%\newcommand{\UPSTItypeDocument}{Custom}
\newcommand{\UPSTIidTypeDocument}{1}

% Titre dans l'en-tête


% Titre dans l'en-tête

\newcommand{\UPSTIvariante}{5}

\newcommand{\UPSTItitreEnTete}{Automatisme industriel}
%\newcommand{\UPSTItitreEnTetePages}{}
\newcommand{\UPSTIsousTitreEnTete}{Introduction aux API}


% Titre
%\newcommand{\UPSTItitrePreambule}{Automatisme industriel}
\newcommand{\UPSTItitre}{La programmation d'un Automate Industriel}

% Durée de l'activité (pour DS, DM et TP)
\newcommand{\UPSTIduree}{3h30}

% Note de bas de première page
%\newcommand{\UPSTInoteBasDePremierePage}{Geoffrey Vaquette}
% Numéro (ajoute " n°1" après DS ou DM)
\newcommand{\UPSTInumero}{2}

% Numéro chapitre
%\newcommand{\UPSTInumeroChapitre}{1}

% En-tête customisé
%\newcommand{\UPSTIenTetePrincipalCustom}{UPSTIenTetePrincipalCustom}

% Message sous le titre
%\newcommand{\UPSTImessage}{Message sous le titre}


% Référence au programme
%\newcommand{\UPSTIprogramme}{\EPBComp \EPBCompP{B1-02}, \EPBCompP{B2-49}, \EPBCompS{B2-50}, \EPBCompS{B2-51}, \EPBCompP{C1-07}, \EPBCompP{C1-08}}

% Si l'auteur n'est pas l'auteur par défaut
%\renewcommand{\UPSTIauteur}{WWOOOOOOWW}

% Si le document est réalisé au nom de l'équipe
%\newcommand{\UPSTIdocumentCollegial}{1}

% Source
\newcommand{\UPSTIsource}{A. Juton, J. Deprez, J. Maillefert, G. Vaquette}

% Version du document
\newcommand{\UPSTInumeroVersion}{1.0}

%-----------------------------------------------
\UPSTIcompileVars		% "Compile" les variables
%%%%%%%%%%%%%%%%%%%%%%%%%%%%%%%%%%%%%%%%%%%%%%%%


% Titre
%\newcommand{\UPSTItitrePreambule}{Automatisme industriel}
\newcommand{\UPSTItitre}{Introduction aux systèmes séquentiels}


\ctikzset{
	logic ports=ieee,
	logic ports/scale=0.7,
}

%%%%%%%%%%%%%%%%%%%%%%%%%%%%%%%%%%%%%%%%%%%%%%%%
% Début du document
%%%%%%%%%%%%%%%%%%%%%%%%%%%%%%%%%%%%%%%%%%%%%%%%
\begin{document}
\UPSTIbuildPage

%\UPSTIobjectif{Durant cette activité, nous allons analyser une trame pour l'envoi d'informations sur une étiquette.}

\tableofcontents

\section{Introduction}
Les équations combinatoires nous permettent de décrire des systèmes dont les actions ne dépendent que de l'état des entrées du système. Cela peut suffire dans certaines applications très simples mais il semble évident que l'introduction de la notion de séquence dans le comportement d'un automate est indispensable à l'élaboration d'un système automatisé. 

\subsection{Les limites du combinatoire}
Un exemple très simple de système séquentiel est un télérupteur : 
\begin{itemize}
	\item Un appui sur l'interrupteur allume la lumière
	\item Un nouvel appui sur l'interrupteur éteint la lumière
\end{itemize}

La sortie dépend de l'état des entrées, mais aussi de son état précédent. En effet, la lumière sera allumé par un appui si elle est éteinte ou par l'absence d'appui si elle est déjà allumée. 

Il est impossible de décrire ce comportement par une équation purement combinatoire.
\pagebreak
\subsection{Automaintien}
\begin{UPSTIactivite}
  \begin{center}
    \begin{tikztimingtable}[
      timing/slope=0,         % no slope
      timing/coldist=2pt,     % column distance
      xscale=5,yscale=2, % scale diagrams
      semithick               % set line width
    ]
  
    S   					&  LHLHHH \\
    R             &  4L  HH \\
    Q             &  LHHHLL \\
  \end{tikztimingtable}%
  \end{center}

  \UPSTIquestion{Tenter d'établir la table de vérité correspondant au chronogramme précédent}
  \begin{UPSTIaCompleterEnv}
    \begin{center}
      \begin{tabular}{cc|c}
        R&S&Q\\\hline
        0&0&?\\
        0&1&1\\
        1&0&0\\
        1&1&0
      \end{tabular}
    \end{center}    
  \end{UPSTIaCompleterEnv}
  \UPSTIquestion{A quel problème êtes-vous confronté ? }
\UPSTIaCompleter{La sortie peut prendre deux états différents pour des combinaisons d'entrées identiques}

	\UPSTIquestion{A partir du chronogramme de l'activité précédente, remplir la table de vérité suivante en prenant en compte la sortie à l'état précédent.}
	\begin{center}
		\begin{tabular}{ccc|cc}
			\hline
			R & S & $Q_{\text{préc}}$ 	& $Q$ 				& Remarque\\ \hline
			0 & 0 & 0 			&\UPSTIcorrection[1]{0}  		& \UPSTIaCompleter{Mémoire}\\
			0 & 0 & 1 			&\UPSTIcorrection[1]{1}			& \UPSTIaCompleter{Mémoire}\\
			0 & 1 & 0 			&\UPSTIcorrection[1]{1} 		& \UPSTIaCompleter{Mise à 1}\\
			0 & 1 & 1 			&\UPSTIcorrection[1]{1}			& \UPSTIaCompleter{Mise à 1}\\\hline
			1 & 0 & 0 			&\UPSTIcorrection[1]{0}  		& \UPSTIaCompleter{Mise à 0}\\
			1 & 0 & 1 			&\UPSTIcorrection[1]{0}			& \UPSTIaCompleter{Mise à 0}\\
			1 & 1 & 0 			&\UPSTIcorrection[1]{0} 		& \UPSTIaCompleter{Mise à 0 prioritaire}\\
			1 & 1 & 1 			&\UPSTIcorrection[1]{0}			& \UPSTIaCompleter{Mise à 0 prioritaire}\\\hline
		\end{tabular}
	\end{center}

\end{UPSTIactivite}

\UPSTIdefinition{\UPSTIaCompleter{Un système séquentiel est un système logique dont le comportement dépend de ses entrées mais aussi \textbf{de son état précédent}}\vspace{2cm}}



\UPSTIremarque[Conséquences]{Une même cause (même combinaison des entrées peut produire des effets différents)}



\section{La bascule RS}


\subsection{Présentation}
\UPSTIdefinition[bascule]{Une bascule (ou verrou) est un circuit logique capable de maintenir les valeurs de ses sorties malgré les changements de valeurs d'entrées. 
Une bascule a donc un effet de mémoire.}

\UPSTIaRetenir{
	\begin{minipage}{.75\textwidth}
		\begin{UPSTIaCompleterEnv}
		Une bascule RS est une bascule à deux entrées de contrôle : $S$ (SET) et $R$ (RESET). Elle possède deux sorties \textit{$Q$} et \textit{$\overline{Q}$}. 
		Son comportement est le suivant : 
		\begin{itemize}
			\item Une mise à 1 de $S$ met la sortie $Q$ à 1
			\item Une mise à 1 de $R$ met la sortie $Q$ à 0
			\item $R=S=0 \rightarrow$ Etat mémoire : la sortie $Q$ maintient son état précédent.
			\item $R$ et $S$ à 1 est un état interdit
		\end{itemize}
		\vspace{1cm}
		\end{UPSTIaCompleterEnv}
		

	\end{minipage}
	\begin{minipage}{.25\textwidth}
		\begin{UPSTIaCompleterEnv}
			\begin{circuitikz}[]
			\draw (0,0) node[flipflop SR](RS){} (RS.bup)
			node[above]{Bascule RS};
			% \draw (RS.pin 1) -- ++(-1,0) node[and port,
			% anchor=out](AND1){}
			% (RS.pin 3) -- ++(-1,0) node[and port,
			% anchor=out](AND2){};
		\end{circuitikz}
		\end{UPSTIaCompleterEnv}
		\centering
	\end{minipage}

\begin{center}
	\begin{tabular}{cccc}
		\hline
		R & S & $Q$ 												& Remarque\\ \hline
		0 & 0 &{$Q$}	 		& {Mémoire}\\
		0 & 1 &{1}						& {Mise à 1}\\
		1 & 0 &0						& {Mise à 0}\\
		1 & 1 &0						& {Interdit}\\\hline
	\end{tabular}	
\end{center}
}

\begin{UPSTIactivite}
\UPSTIquestion{Compléter le chronogramme ci-dessous}
\begin{center}
	\begin{tikztimingtable}[
    timing/slope=0,         % no slope
    timing/coldist=2pt,     % column distance
    xscale=5,yscale=2, % scale diagrams
    semithick               % set line width
  ]

  $R$   					&  				  LLLLLLHLLL                   \\
  $S$                     &  				  LLHLHLLLLH     \\
  $Q$                     &  L \\
  $\overline{Q}$		&    H\\
\end{tikztimingtable}%
\end{center}
\end{UPSTIactivite}


\pagebreak

\begin{UPSTIactivite}[][Bascules ]
	
	\UPSTIquestion{Etablir une bascule RS en langage LADDER}
	\vspace{5cm}
\end{UPSTIactivite}
%\subsection{Réalisation de systèmes avec bascules RS}

% \paragraph{Méthode de travail pour systèmes simples}
% \begin{enumerate}
%   \item Dessiner un chronogramme ou un graphe d'état
%   \item Associer une bascule à chaque sortie
%   \item Rechercher l'événement qui a provoque l'activation de chaque sortie et en déduire l'équation d'activation (SET) associé
%   \item Faire de même pour la désactivation (RESET)
% \end{enumerate}

% \paragraph{Exemple} A partir du chronogramme suivant, établir un logigramme pour coder ce comportement

% \begin{center}
% 	\begin{tikztimingtable}[
%     timing/slope=0,         % no slope
%     timing/coldist=2pt,     % column distance
%     xscale=5,yscale=2, % scale diagrams
%     semithick               % set line width
%   ]

%   BP   					&  LHLLHL \\
%   Q1            &  LHHHLL \\
%   Q2            &  LLHHHL \\
% \end{tikztimingtable}%
% \end{center}


%
\section{Le cycle automate}
\subsection{Introduction aux mémoires d'un automate}
Un automate possède, comme n'importe quel système informatique, de la mémoire lui permettant de stocker et de traiter des données. En pratique, un automate possède plusieurs mémoires différentes ayant chacune une utilité propres.

Dans un premier temps, nous allons considérer trois types de mémoires :
\begin{description}
  \item [Mémoire Entrées : ] Elle contient les informations correspondant à l'état des entrées de l'automate. \textbf{\color{red} Cette mémoire ne peut pas être modifiée par le programme ou l'opérateur.}
  \item [Mémoire interne : ] Elle contient toutes les données utiles au fonctionnement et aux programmes de l'automate.
  \item [Mémoire Sortie : ] Elle contient les informations correspondant à l'état des sorties.
\end{description}
\subsection{Description d'un cycle automate}

L'exécution du programme d'un automate industriel se fait en une succession de cycles que l'on appelle \textbf{cycles automates}.

Chaque cycle peut se décomposer en quatre étapes représentées sur la Figure~\ref{fig:cycleAutomate} et décrites ci-dessous. Ces quatre étapes se succèdent continuellement tant que l'automate n'est pas arrêté par l'opérateur ou par un incident externe.
\begin{description}
  \item[Traitement Interne : ] L'automate effectue des opérations de contrôle et met à jour certains paramètres systèmes (détection des passages en RUN / STOP , mises à jour des valeurs de l'horodateur, ...).
  \item[Lecture des Entrées : ]  L'automate lit les entrées (de façon synchrone) et les recopie dans la mémoire image des entrées.
  \item[Exécution du programme : ] L'automate exécute le programme instruction par instruction et écrit les sorties dans la mémoire image des sorties.
  \item[Ecriture des sorties : ] L'automate bascule les différentes sorties (de façon synchrone) aux positions définies dans la mémoire image des sorties.
\end{description}

\begin{figure}[h!t]
  \centering
  \def\svgwidth{.4\textwidth}
  \import{images/}{cycle_automate.pdf_tex}
  \caption{Déroulement d'un cycle automate}
  \label{fig:cycleAutomate}
\end{figure}

La Figure~\ref{fig:cycleAutomateCapteurToEntree} représente 5 capteurs et leur prise en compte dans la mémoire de l'automate. Pour chacun des capteurs, la prise en compte du passage à l'état 1 ou à l'état 0 ne se fait qu'à l'étape \textbf{I} du cycle automate.

On constate notamment que si le signal observé est trop rapide par rapport au temps de scrutation (cycle automate), il pourrait ne pas être détecté par l'automate (c'est le cas du capteur 3 dans cet exemple). 




\UPSTIaRetenir{Un cycle automate dure environ \SI{10}{ms} (de \SI{1}{ms} à \SI{100}{ms}) durant lesquelles le changement de l'état d'une entrée ne peut pas être pris en compte. Cette durée est appelée \textbf{Temps de scrutation}.}

\UPSTIaRetenir{L'état des entrées et des sorties n'est mis à jour que lors de l'étape \textbf{Lecture des entrées} ou \textbf{Ecriture des sorties}, respectivement. }

\begin{UPSTIactivite}
  \UPSTIquestion{Compléter la Figure~\ref{fig:cycleAutomateCapteurToEntree} en dessinant le signal de la mémoire automate.}
\end{UPSTIactivite}

\begin{figure}[p]
  \centering
  \def\svgwidth{.9\textwidth}
  \import{images/}{cycleAutomateTemporel.pdf_tex}
  \caption{Déroulement d'un cycle automate}
  \label{fig:cycleAutomateCapteurToEntree}
\end{figure}



\end{document}

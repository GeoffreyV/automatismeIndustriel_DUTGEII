La première chose à vérifier avant d'entreprendre la programmation d'un automate est de vérifier le bon fonctionnement des entrées et sorties de l'automate.
\begin{UPSTIactivite}[][Test des capteurs]
	Avec l'automate sous tension, vérifier \textbf{un à un} le bon fonctionnement de \textbf{tous les capteurs} en vérifiant que la LED correspondante sur le module d'entrées s'allume ainsi que le changement d'état dans la table d'animation du projet.

	Si un capteur ne fonctionne pas ou que son état ne varie pas dans la table d'animation, chercher alors la cause de ce disfonctionnement.
\end{UPSTIactivite}
\UPSTIboiteGenerique{Aide à la rédaction}{\bcplume}{
A titre d'exemple, pour la présentation des tests des capteurs dans votre compte-rendu, vous pouvez expliquer la démarche générale puis insérer une capture d'écran du test d'un des capteurs avec l'explication associée. Il n'est pas alors nécessaire de faire une capture pour chaque capteur.

Précisez également s'il s'agit d'une structure locale ou déportée et décrivez tout disfonctionnement rencontré et comment il a été corrigé.}

\begin{UPSTIactivite}[][Test des actionneurs]
Pour tester les actionneurs, il est nécessaire de commander les sorties de l’automate.

Dans la table d'animation, cliquer sur \textit{Modifications} afin d'activer la commande des sorties.

Vérifier \textbf{un à un} le bon fonctionnement de \textbf{tous les actionneurs} en vérifiant que la LED correspondante sur le module de sortie s'allume et que l'actionneur s'active.
\end{UPSTIactivite}

Vous rédigerez un compte-rendu détaillé des manipulations effectuées celui du TP 3 servira d'entraînement et comptera avec un coefficient moins important que celui du TP4.

\UPSTIremarque{
Le compte-rendu évalue votre capacité à \textbf{expliquer et synthétiser} votre démarche et les manipulations effectuées.
Les manipulations en elle-même sont observées durant la séance par l'enseignant.}

Vous pourrez donc insérer des captures d'écrans, photos et tout schéma pouvant aider à la compréhension de votre propos.
Un bon compte-rendu est un compte-rendu \textbf{lisible et entièrement compréhensible} par une personne n'ayant pas participé au TP et ayant un niveau de connaissance similaire au vôtre.

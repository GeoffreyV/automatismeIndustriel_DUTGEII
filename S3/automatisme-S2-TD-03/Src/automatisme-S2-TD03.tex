%%%%%%%%%%%%%%%%%%%%%%%%%%%%%%%%%%%%%%%%%%%%%%%%
% E.Pinault-Bigeard - e.pinault-bigeard@upsti.fr
% http://s2i.pinault-bigeard.com
% CC BY-NC-SA 2.0 FR - http://creativecommons.org/licenses/by-nc-sa/2.0/fr/
%%%%%%%%%%%%%%%%%%%%%%%%%%%%%%%%%%%%%%%%%%%%%%%%
\documentclass[11pt, multicol]{article}
%%%%%%%%%%%%%%%%%%%%%%%%%%%%%%%%%%%%%%%%%%%%%%%%
% Package UPSTI_Document
%%%%%%%%%%%%%%%%%%%%%%%%%%%%%%%%%%%%%%%%%%%%%%%%
%%%%%%%%%%%%%%%%%%%%%%%%%%%%%%%%%%%%%%%%%%%%%%%%
% Package UPSTI_Document
%%%%%%%%%%%%%%%%%%%%%%%%%%%%%%%%%%%%%%%%%%%%%%%%
\usepackage{subcaption}
\usepackage[usenames, svgnames, dvipsnames]{xcolor}
\usepackage{UPSTI_Document}
\usepackage{pgfplots}
\definecolor{darkspringgreen}{rgb}{0.09, 0.45, 0.27}

\newcommandx*{\dessinRepereFigGeo}[5][1=\vx{},2=\vy{},3=\vz{},4=,5=0]
	{
		\draw [->,very thick] (0,0) -- (1,0) ;
		\draw [->,very thick] (0,0) -- (0,1) ;
    \fill[white] (0,0) circle (0.13);
    \draw [->,very thick] (0,0) circle (0.13);
    \ifnumequal{#5}{0} {% z vers nous
      \fill[black] (0,0) circle (0.03);
      \draw [->,thick] (0,0) circle (0.04);
    }{% z vers la feuille
  		\begin{scope} [rotate=45]
  			\draw [-,thick] (0,-0.12) -- (0,0.12) ;
  			\draw [-,thick] (-0.12,0) -- (0.12,0) ;
  		\end{scope}
    }
		\draw [anchor=north west] (1.1,0) node {${#1}$};
		\draw [anchor=south west] (0,1.1) node {${#2}$};
		\draw [anchor=north east] (-0.1,0) node {${#3}$};
		\draw [anchor=north west] (-0.1,-0.1) node {${#4}$};
	}

	\usepackage{array}
	\newcolumntype{L}[1]{>{\raggedright\let\newline\\\arraybackslash\hspace{0pt}}m{#1}}
	\newcolumntype{C}[1]{>{\centering\let\newline\\\arraybackslash\hspace{0pt}}m{#1}}
	\newcolumntype{R}[1]{>{\raggedleft\let\newline\\\arraybackslash\hspace{0pt}}m{#1}}

	\usepackage{pifont}% http://ctan.org/pkg/pifont
\newcommand{\cmark}{\color{green}\ding{51}}%
\newcommand{\xmark}{\color{red}\ding{55}}%
\newcommand{\fmark}{\ding{229}}%
\newcommand{\itemc}{\item[\cmark]}%
\newcommand{\itemx}{\item[\xmark]}%
\newcommand{\itemf}{\item[\fmark]}%

%---------------------------------%
% Paramètres du package
%---------------------------------%

% Version du document (pour la compilation)
% 1: Document prof
% 2: Document élève
% 3: Document à publier
\input{variables/documentVersion}

% Classe
% 1: PTSI				6: PSI*			11: TSI2		16: Spé
% 2: PT	(par défaut)	7: MPSI			12: ATS
% 3: PT*				8: MP			13: PC
% 4: PCSI				9: MP*			14: PC*
% 5: PSI				10: TSI1		15: Sup
\newcommand{\UPSTIidClasse}{0}
\newcommand{\UPSTIclasse}{S2}


% Matière
% 1: S2I (par défaut)    2: IPT     3: TIPE
% 6: Vie au lycée
\newcommand{\UPSTIidMatiere}{0}
\newcommand{\UPSTIintituleMatiere}{Automatisme}
\newcommand{\UPSTIsigleMatiere}{Autom}
% Type de document
% 0: Custom*				7: Fiche Métho de			14: Document Réponses
% 1: Cours (par défaut)		8: Fiche Synthèse    		15: Programme de colle
% 2: TD     				9: Formulaire
% 3: TP						10: Memo
% 4: Colle					11: Dossier Technique
% 5: DS						12: Dossier Ressource
% 6: DM						13: Concours Blanc
% * Si on met la valeur 0, il faut décommenter la ligne suivante:
%\newcommand{\UPSTItypeDocument}{Custom}
\newcommand{\UPSTIidTypeDocument}{1}

% Titre dans l'en-tête


% Titre dans l'en-tête

\newcommand{\UPSTIvariante}{5}

\newcommand{\UPSTItitreEnTete}{Automatisme industriel}
%\newcommand{\UPSTItitreEnTetePages}{}
\newcommand{\UPSTIsousTitreEnTete}{Introduction aux API}



% Durée de l'activité (pour DS, DM et TP)
\newcommand{\UPSTIduree}{1h}

% Note de bas de première page
%\newcommand{\UPSTInoteBasDePremierePage}{Geoffrey Vaquette}
% Numéro (ajoute " n°1" après DS ou DM)
\newcommand{\UPSTInumero}{3}

% Numéro chapitre
%\newcommand{\UPSTInumeroChapitre}{1}

% En-tête customisé
%\newcommand{\UPSTIenTetePrincipalCustom}{UPSTIenTetePrincipalCustom}

% Message sous le titre
%\newcommand{\UPSTImessage}{Message sous le titre}


% Référence au programme
%\newcommand{\UPSTIprogramme}{\EPBComp \EPBCompP{B1-02}, \EPBCompP{B2-49}, \EPBCompS{B2-50}, \EPBCompS{B2-51}, \EPBCompP{C1-07}, \EPBCompP{C1-08}}

% Si l'auteur n'est pas l'auteur par défaut
%\renewcommand{\UPSTIauteur}{WWOOOOOOWW}

% Si le document est réalisé au nom de l'équipe
%\newcommand{\UPSTIdocumentCollegial}{1}

% Source
\newcommand{\UPSTIsource}{G. Vaquette, A. Juton, J. Deprez, J. Maillefert}

% Version du document
\newcommand{\UPSTInumeroVersion}{1.0}

%-----------------------------------------------
\UPSTIcompileVars		% "Compile" les variables
%%%%%%%%%%%%%%%%%%%%%%%%%%%%%%%%%%%%%%%%%%%%%%%%


%%%%%%%%%%%%%%%%%%%%%%%%%%%%%%%%%%%%%%%%%%%%%%%%
% Début du document
%%%%%%%%%%%%%%%%%%%%%%%%%%%%%%%%%%%%%%%%%%%%%%%%
\usepackage{grafcet}
\usetikzlibrary[circuits.plc.ladder]            %     
\newlength{\ladderskip}\setlength{\ladderskip}{5\tikzcircuitssizeunit}%5\tikzcircuitssizeunit    = 355pt
\newlength{\ladderrungsep}
\setlength{\ladderrungsep}{.2\ladderskip}
\def\ladderrungend#1{\pgftransformyshift{-#1\ladderskip-\ladderrungsep}}

\usepackage{enumitem}
\begin{document}
\UPSTIbuildPage

%\UPSTIobjectif{Durant cette activité, nous allons analyser une trame pour l'envoi d'informations sur une étiquette.}

%\tableofcontents
\section{Ai-je bien compris le cours ? }
\UPSTIquestion{Compléter les phrases suivantes : }
\begin{itemize}
  \item Dans un programme automate, l’association des symboles à une adresse est faite dans \UPSTIcorrection{la table des symboles}
  \item Un programme automate est organisé en \UPSTIcorrection{tâches ou sections}

  \item Chaque tâche est exécutée instruction par instruction, c’est-à-dire de façon \UPSTIcorrection{séquentielle}

  \item Le cycle de l’automate dure \UPSTIcorrection{de 1 à 100ms}

\end{itemize}

\UPSTIquestion{Choisir la bonne réponse pour chaque proposition suivante : }
\begin{itemize}
  \item Une variable automate de type BOOL est (TOR / un nombre)
  \item Une variable automate de type INT est (TOR / un nombre)
  \item La variable automate dont l’adresse est \%I0.1.6 est (une entrée/une sortie/interne à l’automate)
  \item La variable automate dont l’adresse est \%Q0.2.4 est (une entrée/une sortie/interne à l’automate)
  \item La variable automate dont l’adresse est \%M4 est (une entrée/une sortie/interne à l’automate)
\end{itemize}

\UPSTIquestion{L’une de ces problématiques est séquentielle, trouver laquelle}

  \begin{itemize}
  \item Le voyant s’allume si le capteur d’usure est à 1
  \item Les conditions de démarrage sont réunies si le capot est fermé et le bain d’huile est à la bonne température
  \item Quand la cuve a atteint la bonne température, les produits doivent être ajoutés puis mélangés
  \item L’alimentation est coupée si une présence est détectée dans le champ opératoire
  \end{itemize}
  \UPSTIcorrection{Quand la cuve a atteint la bonne température, les produits doivent être ajoutés puis mélangés}

\UPSTIquestion{Les activités 1- 2- 3- ci-dessous concernent le domaine des automatismes industriels. Associer à chacune le
bon mot clé a- b- c-}

\begin{minipage}{.48\textwidth}
  \begin{enumerate}
    \item une fabrication agroalimentaire
    \item la mise en sécurité d’un ilot robotisé
    \item une centrale de traitement d’air
  \end{enumerate}
\end{minipage}
\begin{minipage}{.48\textwidth}
  \begin{enumerate}[label=\alph*]
    \item Régulation
    \item Automatisme combinatoire
    \item Automatisme séquentiel
  \end{enumerate}
\end{minipage}


\UPSTIquestion{La fonction mémoire (SET ou RESET) d'un LADDER set à : (Trouver la bonne proposition)}
\begin{itemize}
  \item Suite à un ordre, elle fournit une impulsion sur un signal
  \item Suite à un ordre, elle permet de maintenir le niveau 1 ou 0 d’un signal
  \item Elle réalise la fonction SIGNAL = ORDRE
\end{itemize}



\section{Un tableau automatisé}
On souhaite réaliser la commande numérique d’un tableau dans une salle de classe (système qui fut disponible
dans les amphithéâtres de l’IUT de Cachan).

Le tableau se déplace le long de deux glissières grâce à un moteur. Le déplacement du tableau de haut en bas ou
de bas en haut est lié au sens de rotation de ce moteur. Le sens de rotation du moteur est déterminé par l’état de
deux signaux logiques MotM et MotD tels que

\begin{table}[hb]
\centering
  \begin{tabular}{c|c|c}
    MotM & MotD &\\
    0     & 0   & Tableau arrêté\\
    0&1&Descente du tableau\\
    1&0&Montée du tableau
  \end{tabular}
  \caption{Comportement du tableau selon l'activation des moteurs}
  \label{tab:sortieTableau}
\end{table}

Un capteur de fin de course FcH fournit un signal à ‘1’ si le tableau se situe tout en haut de la glissière.

Un capteur de fin de course FcB fournit un signal à ‘1’ si le tableau se situe tout en bas de la glissière.

L’utilisateur (le prof ...) dispose de deux boutons-poussoirs BpM, BpD (« montée », « descente ») pour régler la
position du tableau. Ces boutons-poussoirs fournissent une impulsion à ‘1’ lorsqu’on les enfonce.


\UPSTIquestion{Identifier les entrées et les sorties de la commande de ce système}

\UPSTIcorrection{\begin{description}
  \item [Entrées : ] FcH, FCb, BpM, BbD
  \item [Sorties : ]  MotM, MotD
\end{description}}



\UPSTIquestion{Représenter le système sous forme d'une boite comportant les entrées et sorties}
\UPSTIquestion{Le comportement décrit est-il combinatoire ou séquentiel ?}

\UPSTIcorrection{Combinatoire (la sortie ne dépend que des entrées)}

\UPSTIquestion{Proposer un programm en LADDER respectant le cahier des charges.}

\UPSTIcorrection{\begin{tikzpicture}[circuit plc ladder, thick, x=\ladderskip, y=\ladderskip]
  \draw(0,0) to [contact NO={info={BpM}}] ++ (1,0)
    to [contact NC={info={FcH}}] ++ (1,0) -- ++(1,0)
    to [coil={info={Sortie}}] + (1,0) coordinate(laddertopright);
    \ladderrungend{1}
    \draw let \p1=(laddertopright) in
    (0,   \y1+0.7\ladderskip) -- (0,    \ladderskip)
    (\x1, \y1+0.7\ladderskip) -- (\x1,  \ladderskip);

  \draw(0,-.2) to [contact NO={info={BpD}}] ++ (1,0)
    to [contact NC={info={FcB}}] ++ (1,0) -- ++(1,0)
    to [coil={info={Sortie}}] + (1,0) coordinate(laddertopright);
    \ladderrungend{1.5}
    \draw let \p1=(laddertopright) in
    (0,   \y1+0.7\ladderskip) -- (0,    \ladderskip)
    (\x1, \y1+0.7\ladderskip) -- (\x1,  \ladderskip);
\end{tikzpicture}
}


\section{Pupitre simple (TP 1 et 2)}
On reprend l'enoncé du pupitre, TD 1.
\resetNumQuestion
\UPSTIquestion{Ecrire le programme en LADDER tel que le voyant V0 s’allume si et seulement si on appuie sur le bouton BP0 ou sur le bouton BP1}
\UPSTIquestion{Ecrire le programme répondant au cahier des charges suivant : le bouton BP2 enfoncé allume V0 et V3, le bouton BP1 enfoncé allume V0 et V2. Il y a ici 3 sorties, le programme doit donc comporter 3 réseaux.}
\UPSTIquestion{Auto-maintient : écrire un programme en LADDER tel que l’appui sur BP0 allume V0, l’appui sur BP1 éteint V0. S’agit-il d’un programme combinatoire ou séquentiel ? Cette structure doit être connue.}
\UPSTIquestion{Gestion d’un compteur : Ecrire le programme répondant au cahier des charges suivant : Le compteur s'incrémente de 1 à chaque appui sur BP2. Il est remis à zéro par BP0. Quand la valeur du compteur est strictement supérieure à 3, le voyant V1 s’allume. « compteur » est une variable interne de de type INT.}
\end{document}
